% !TEX encoding = UTF-8
% !TEX TS-program = pdflatex
% !TEX root = ../Tesi.tex
% !TEX spellcheck = it-IT

%*******************************************************
% Introduzione
%*******************************************************
%\cleardoublepage
\pdfbookmark{Introduzione}{introduzione}

\chapter*{Introduzione}

In un epoca in cui le alte prestazioni sono drasticamente
essenziali nei pi� disparati campi scientifici, ci troviamo spesso a dover
fronteggiare problematiche di inefficienza o esigenze di miglioramento con
molteplici mezzi e soluzioni presenti nelle teorie informatiche moderne. Il
termine \textbf{velocit�} � diventato sinonimo di successo in diversi contesti
ed � il protagonista principale di molti obiettivi progettuali dei nostri tempi.
Tra i tanti fattori che comportano il successo di un calcolatore, di un software
o di un'applicazione per dispositivi mobili possiamo distinguere naturalmente
anche la velocit� di risposta.  

Sin dai primi calcolatori, le migliorie apportate alle macchine furono
progettate e implementate quasi sempre con lo scopo di incrementare la
la velocit�. Negli ultimi vent'anni in particolar modo l'aumento
prestazionale � stato e continua ad essere un esigenza. 
Dal punto di vista hardware c'� stata una vera rivoluzione
che nel corso degli anni ha portato ad avere le pi� innovative tecnologie
apportate ai processori che oggi popolano i nostri calcolatori. La richiesta
prestazionale ha inciso in maniera dirompente nel mercato dello sviluppo
del software portando cos� alla progettazione di tecniche innovative
per migliorare sempre di pi� l'esperienza utente e le performance.
Proprio i miglioramenti apportati hanno stravolto l'esperienza di utilizzo
giornaliero integrando sempre di pi� l'utilizzo dei computer e altri dispositivi
nella vita di tutti i giorni. Un esempio banale potrebbe essere un'applicazione
mobile che ha il compito di fornire informazioni ai cittadini relative ai mezzi
pubblici di trasporto. Se l'applicazione ha un tempo di risposta molto lenta,
dovuta per esempio alla mole di dati da dover processare, potrebbe risultare
inutilizzabile. 

Ecco perch� oggi il calcolo parallelo � molto utilizzato nello
sviluppo del software. I notevoli miglioramenti lato hardware, che hanno
dato i natali ai processori di ultima generazione dotati di pi� core
(i cos�detti \textit{multicore}), hanno comportato lo sviluppo di tecniche
innovative proprio per poter utilizzare a pieno questa nuova tipologia di
processori. Il calcolo parallelo oggi ha un grosso impatto in diverse aree
informatiche, dalla ricerca scientifica fino allo sviluppo di software
commerciale. Grazie a questa branca	informatica oggi � possibile sfruttare a
pieno le nuove tecnologie e tutti i vantaggi dei nuovi dispositivi ultraveloci.
Nel corso degli anni sono stati sviluppati e migliorati anche diverse
metodologie e modelli paralleli, a partire dal miglioramento delle architetture
all'arrivo dei nuovi framework. I due modelli di architettura pi� utilizzati
oggi si dividono in modelli basati su memoria condivisa e modelli basati sul
paradigma dello scambio di messaggi.

La GPU (Graphics Processing Unit), inizialmente utilizzata solamente per il
rendering grafico delle immagini, negli ultimi anni � stata scoperta come
potenza di calcolo con velocit� teoriche quasi dieci volte superiori alle
normali CPU. Questo ha rivoluzionato le teorie sul calcolo parallelo che in
particolare ha tratto vantaggi anche in termini di costi. Dato l'iniziale
successo delle GPU si � introdotto nel parallel computing il termine GPGPU
programming, che rappresenta tutti gli utilizzi delle GPU che non comprendano il
rendering grafico. Oggi la GPGPU programming � utilizzata in decine di campi
scientifici, dalla bio-informatica all'analisi finanziaria, coprendo anche campi
come la fluidodinamica computazionale, l'apprendimento automatico e la scienza
dei dati \cite{CUDA_APP:2015}. 

Nvidia Corporation, societ� che opera nello sviluppo delle GPU ormai da
tantissimi anni, ha puntato molto sul calcolo parallelo producendo device sempre
pi� adatti per uno scopo computazionale. \textbf{CUDA} � un architettura
completa (hardware e software) creata proprio da Nvidia che abilita alla GPGPU
programming sfruttando proprio le schede grafiche Nvidia oramai molto diffuse.
In particolare sin dal 2007, anno del suo lancio, CUDA ha subito numerosi
aggiornamenti in cui ognuno di loro ha portato diverse features innovative che
rendono la GPGPU programming sempre pi� semplice ed estremamente comoda. L'unico
neo di questa potentissima architettura � la portabilit�. Non � infatti
possibile eseguire programmi scritti in CUDA su schede video diverse dalla
Nvidia. Per questo � stato creato nel 2008 OpenCL che risulta per�
leggermente diverso sia nel suo utilizzo che nella sua architettura.

Tra i sistemi complessi maggiormente noti in campo scientifico possiamo trovare
gli \textbf{automi cellulari}. Gli automi cellulari sono insiemi di regole
logico-matematiche capaci di descrivere sistemi complessi e rappresentarne la
loro evoluzione nel tempo. Un AC pu� essere descritto come uno spazio suddiviso
in celle regolari ognuna delle quali pu� trovarsi in un numero finito di stati.
La legge che detta la sua evoluzione nel tempo � chiamata funzione di
transizione comune per tutte le celle. Uno degli input per ogni cella �
l'insieme dei suoi vicini configurati da una relazione di vicinanza che non
varia nel tempo e nello spazio \cite{Dambrosio:2003}. 

L'applicazione degli automi cellulari trova spazio in diversi campi di ricerca
come la simulazione del comportamento dei pedoni nei centri
commerciali \cite{Jarek:2014}, fino alla simulazioni di fenomeni naturali come
frane \cite{SCIDDICA:1999} e colate laviche \cite{SCIARA:2004}.

OpenCAL (Open Cellular Automata Library) � una libreria open source per la
modellazione e la simulazione di modelli basati su automi cellulari, in
particolare su automi cellulari complessi (CCA). La libreria nasce per rendere
l'implementazione degli automi cellulari pi� semplice e immediata. Infatti
grazie al suo utilizzo, l'utente potr� concentrarsi completamente sulla
progettazione del modello senza dover dare particolari attenzioni ai dettagli
implementativi. Nel corso del tempo sono state sviluppate diverse versioni della
libreria utilizzando il calcolo parallelo, dalla versione in OpenMP alle
versioni OpenCL e CUDA. Grazie alla sua comodit� e le sue performance pu�
essere considerata una valida alternativa a software per la creazione di
modelli basati su automi cellulari come Camelot.

Questo lavoro di tesi ha comportato la progettazione e la successiva
implementazione di \textbf{OpenCAL-CUDA}. E' stata utilizzata
l'architettura CUDA per la progettazione di un ulteriore versione parallela
della libreria sfruttando la GPGPU programming.

La tesi � suddivisa come segue:
\ref{cap:O}
\begin{description}
\item[{\hyperref[cap:Il calcolo parallelo]{Il primo capitolo}}]
offre una visione generale sul parallel computing. Approfondiremo l'argomento
con la descrizione delle pi� famose tecniche e metodologie attualmente
utilizzate, compresi i modelli di comunicazione e un esempio di progettazione
di un algoritmo parallelo. Ci sar� un breve cenno sul paradigma
di programmazione OpenMP seguito da una parte introduttiva della GPGPU
programming.

\item[{\hyperref[cap:CUDA]{Il secondo capitolo}}]
descrive l'architettura CUDA, utilizzata nel lavoro di tesi. In particolare,
dopo una breve introduzione, verr� descritta l'architettura hardware e la sua
interfaccia di programmazione. Nello stesso capitolo verranno introdotti i tools
di sviluppo messi a disposizione per operare con questo tipo di archietettura.

\item[{\hyperref[cap:Automi Cellulari]{Il terzo capitolo}}]
introduce la teoria basata sugli automi cellulari e in particolare nel paragrafo
\ref{par:CCA} si introdurranno gli automi cellulari complessi.

\item[{\hyperref[cap:OpenCAL]{Il quarto capitolo}}]
introduce la libreria OpenCAL descrivendo in dettaglio la definizione di un
modello e di una simulazione, in particolare verranno proposti degli esempi di
utilizzo.

Infine,
\item[{\hyperref[cap:OpenCAL-CUDA]{Il quinto capitolo}}]
dopo una breve introduzione, descrive le scelte progettuali intraprese
nella parallelizzazione della versione CUDA di OpenCAL. Nei paragrafi
successivi, si descriver� la nuova forma strutturale che rappresenta questa
nuova implementazione e infine, come per il quarto capitolo, ci saranno degli
esempi di utilizzo abbinati ad alcuni test effettuati per validare il progetto
svolto e gli obiettivi di tesi.

\end{description}



\newpage

\null 

\thispagestyle{empty} 

\newpage