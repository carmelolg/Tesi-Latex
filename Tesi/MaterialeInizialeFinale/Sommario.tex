% !TEX encoding = UTF-8
% !TEX TS-program = pdflatex
% !TEX root = ../Tesi.tex
% !TEX spellcheck = it-IT

%*******************************************************
% Sommario+Abstract
%*******************************************************
\cleardoublepage
\phantomsection
\pdfbookmark{Sommario}{Sommario}
\begingroup
\let\clearpage\relax
\let\cleardoublepage\relax
\let\cleardoublepage\relax

\chapter*{Sommario}


\selectlanguage{italian}

In questo lavoro di tesi ho progettato e implementato una versione parallela della libreria per automi cellulari OpenCAL.\\
OpenCAL, si propone di facilitare l'implementazione di sistemi complessi basati
su automi cellulari, offrendo funzionalit� complete per progettare un modello
e simulare la sua evoluzione nel tempo.
Il mio lavoro � consistito nella progettazione, e successiva implementazione,
della parallelizzazione di OpenCAL utilizzando le schede grafiche per il
calcolo general-purpose (General Purpose Computation with Graphics Processing
Units - GPGPU), adottando il Compute Unified Device Architecture (CUDA)
framework di NVIDIA con lo scopo di migliorare le performance.

\textbf {TODO}
\begin{itemize}
\item Risultati ottenuti
\item validità della gpgpu programming
\end{itemize}

\endgroup			