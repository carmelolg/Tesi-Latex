% !TEX encoding = UTF-8
% !TEX TS-program = pdflatex
% !TEX root = ../Tesi.tex
% !TEX spellcheck = it-IT

%************************************************
\chapter{Conclusioni}
\label{cap:Conclusioni}
%************************************************

Gi� da molto tempo l'approccio sistematico al parallel computing ha comportato
miglioramenti generali nell'utilizzo dei sistemi informatici. La ricerca basata
sull'incremento delle performance dei moderni computer e il calcolo ad alte
prestazioni ha trovato campo fertile in numerosi settori dell'informatica tra i
quali la modellistica e la simulazione.

L'obiettivo di questo lavoro di tesi � stata la parallelizzazione della libreria
per lo sviluppo di modelli basati su automi cellulari OpenCAL. Gli automi
cellulari per loro natura si prestano egregiamente ad un approccio parallelo,
proprio per questo � stata immediata la scelta del parallel computing per
migliorare le performance della libreria OpenCAL. La versione parallela
OpenCAL-CUDA, come si intuisce, � stata implementata tramite l'archiettura CUDA
sviluppata e rilasciata dalla societ� NVIDIA Corporation. In particolare � stato
utilizzato il linguaggio CUDA C, estensione del linguaggio C, per
l'implementazione del codice parallelo.

Tutte le caratteristiche appartenenti ad OpenCAL sono state mantenute, tuttavia 
l'implementazione dei modelli e dei loro processi elementari sono state
adattate al tipo di architettura utilizzata. In particolare sono presenti alcuni
cambiamenti dovuti ad una filosofia implementativa diversa tra l'approccio
sequenziale e quello parallelo.

NVIDIA dal 2006 ai giorni nostri, ha rilasciato in maniera
frequente aggiornamenti per l'architettura CUDA con numerosi
miglioramenti relativi alla leggibilit� del codice e alle performance. Le API di
CUDA compatibili con i device NVIDIA hanno consentito la realizzazione
del progetto.

Gli automi cellulari, come spiegato nel capitolo \ref{cap:Automi Cellulari},
evolvono basandosi sulla funzione di transizione, in particolare per gli automi
cellulari complessi (CCA, \ref{par:CCA}) l'evoluzione dipende da pi� processi
elementari. Questa funzione di transizione viene eseguita allo stesso modo su
ogni cella dello spazio cellulare. Questo tipo di approccio viaggia in
perfetta sintonia con la filosofia del parallel computing.

In OpenCAL-CUDA vengono creati un numero di blocchi e thread in base al numero
di celle dello spazio cellulare in modo tale da assegnare un thread per ogni
cella. In questo modo, tutti i thread eseguono la stessa operazione nello stesso
momento su celle diverse incrementando le performance e minimizzando i tempi di
risposta. Per l'implementazione di un automa cellulare si pu� utilizzare anche
l'ottimizzazione delle celle attive. Questo approccio, che utilizza solamente le
celle attive escludendo le celle in stato quiescente, � supportato dalla
versione parallela grazie all'utilizzo della stream compaction (par.
\ref{par:streamcompaction}). La stream compaction ha il compito di elaborare e
comprimere i dati sparsi. I dati sparsi nel nostro caso sono il numero di celle
attive ad un determinato tempo $t$. Si istanzieranno dunque un determinato
numero di thread in base al numero di celle effettivamente attive.

Dopo aver terminato la parallelizzazione di OpenCAL sono stati implementati
diversi modelli con il fine di testare il lavoro di tesi. Tra i vari modelli
implementati, quello utilizzato per confrontare i tempi di esecuzione e i vari
miglioramenti di performance � stato SCIARA \cite{SCIARA:2001}
\cite{SCIARA:2004}.
I test eseguiti si sono basati sull'implementazione del modello SCIARA sia con
l'ottimizzazione delle celle attive che senza alcun tipo di ottimizzazione. 

Per raccogliere i dati relativi ai tempi di esecuzione per la versione parallela
del modello � stata utilizzata la super-macchina ``Stromboli'' situata al centro
di calcolo ad alte prestazioni dell'Unical, dotata di un processore Quad Xeon da
2.8GHz. Per quanto riguarda le schede grafiche utilizzate sono state scelte due
differenti schede: la prima � una Tesla K20c, la seconda una GeForce GT750M
entrambe di marca NVIDIA.

Con OpenCAL-CUDA per la versione implementata senza ottimizzazioni raggiunge
una speedup di $\sim 29\times$ in media, utilizzando la scheda Tesla K20c. Per
quanto riguarda i test sulla versione con l'ottimizzazione delle celle attive
con 200 crateri raggiunge una speedup di $\sim10\times$ utilizzando la scheda
GeForce GT 750M.

Oggi OpenCAL � un progetto open source avviato. Sono presenti anche diverse
implementazioni della libreria tra cui due versioni parallelizzate utilizzando
i linguaggi di programmazione OpenCL e OpenMP. Un ulteriore versione della
libreria integra OpenGL per la visualizzazione grafica. 

Un possibile sviluppo futuro di OpenCAL-CUDA potrebbe essere l'implementazione
della versione 3D, mentre a scopo statistico e di ricerca sarebbe sicuramente
interessante effettuare un confronto delle performance tra le varie
implementazioni parallelizzate.
