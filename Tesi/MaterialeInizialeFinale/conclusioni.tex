% !TEX encoding = UTF-8
% !TEX TS-program = pdflatex
% !TEX root = ../Tesi.tex
% !TEX spellcheck = it-IT

%************************************************
\chapter{Conclusioni}
\label{cap:Conclusioni}
%************************************************

Gi� da molto tempo l'approccio sistematico al parallel computing ha comportato
miglioramenti generali nell'utilizzo dei sistemi informatici. La ricerca basata
sull'incremento delle performance ha trovato campo fertile in numerosi
settori dell'informatica tra i quali la modellistica e la simulazione. 

L'obiettivo di questo lavoro di tesi � stata la parallelizzazione della libreria
per lo sviluppo di modelli basati su automi cellulari OpenCAL.Gli automi
cellulari si prestano bene ad un approccio parallelo, proprio per questo � stata
immediata la scelta del parallel computing per migliorare le performance della
libreria OpenCAL. La versione parallela OpenCAL-CUDA, come si intuisce, � stata
implementata tramite l'archiettura CUDA sviluppata e rilasciata dalla societ�
NVIDIA Corporation. In particolare � stato utilizzato il linguaggio CUDA C,
estensione del linguaggio C, per l'implementazione del codice parallelo.

NVIDIA dal 2006 ai giorni nostri, ha rilasciato in maniera
frequente aggiornamenti relativi all'architettura CUDA con numerosi
miglioramenti relativi alla leggibilit� del codice e alle performance. Le API di
CUDA compatibili con i device NVIDIA hanno consentito la realizzazione
del progetto.

Gli automi cellulari, come spiegato nel capitolo \ref{cap:Automi Cellulari},
evolvono basandosi sulla funzione di transizione, in particolare per gli automi
cellulari complessi (CCA, \ref{par:CCA}) l'evoluzione dipende da pi� processi
elementari. Questa funzione di transizione viene eseguita allo stesso modo su
ogni cella dello spazio cellulare. Dunque questo approccio si presta bene al
calcolo parallelo.

In OpenCAL-CUDA vengono creati un numero di blocchi e thread in base al numero
di celle dello spazio cellulare in modo da avere un thread per ogni cella.
In questo modo, tutti i thread eseguono la stessa operazione nello stesso
momento su celle diverse incrementando inesorabilmente le performance. 
Per l'implementazione di un automa cellulare si pu� utilizzare anche
l'ottimizzazione delle celle attive. Questo approccio, che utilizza solamente le
celle attive escludendo le celle in stato quiescente, � supportato dalla
versione parallela adattando dunque il lavoro dei thread all'utilizzo delle
celle attive utilizzando la stream compaction (par. \ref{par:streamcompaction}).
La stream compaction ha il compito di elaborare e comprimere i dati sparsi. 

Con OpenCAL-CUDA � possibile creare applicazioni parallele per modelli basati su
automi cellulari, in particolare automi cellulari complessi. Tutte le
caratteristiche appartenenti ad OpenCAL sono state mantenute, tuttavia 
l'implementazione dei modelli e dei loro processi elementari � leggermente
cambiata tra le due versioni della libreria. I cambiamenti son dovuti alla
differente architettura utilizzata, oltre che una filosofia
implementativa diversa.

Per poter validare..
